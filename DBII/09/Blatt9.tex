\documentclass{uni_tue_template}

\usepackage{color}
\usepackage[colorlinks]{hyperref}
\definecolor{darkred}{rgb}{0.5,0,0}
\definecolor{darkgreen}{rgb}{0,0.5,0}
\definecolor{darkblue}{rgb}{0,0,0.7}
\hypersetup{
			linkcolor={darkblue},
			filecolor={darkgreen},
			urlcolor={darkred},
			citecolor={darkblue},
}

\usepackage[export]{adjustbox}
\usepackage{tikz}
\usetikzlibrary{shapes}
\usepackage{rotating}
\usepackage{sistyle}

\lstdefinestyle{sql}{%
	language=SQL,%
	commentstyle=\color{comment},%
	stringstyle=\color{string},%
	keywordstyle=\color{keyword}\bfseries,%
	morekeywords={text, real, begin},%
	basicstyle=\ttfamily\footnotesize,%
	numbers=left,
	numberstyle=\tiny\color{black},
	stepnumber=1,
	showstringspaces=true,%
	columns=fixed,%
	moredelim=[is][\itshape]{@@}{@@},%
	tabsize=2
}
\lstset{aboveskip=-.6cm,style=sql, firstline=4}

\newcommand{\otext}[1]{\overset{\mathclap{\rule[-.3\baselineskip]{0pt}{0cm}\textmd{#1}}}}
\newcommand{\set}[1]{\mathbb{#1}}
\newcommand{\code}[1]{\texttt{{\footnotesize #1}}}
\newcommand{\md}[1]{\textmd{#1}}
\newcommand{\tn}[1]{\textup{\textsf{#1}}}
\newcommand{\ti}[1]{\textit{\textsf{#1}}}

% content of left head area e.g. subject like ETI
\def \headLeft{Andreas Schmied (3087156),\newline Tobias Stumpp (3798377)}

% content of center head area, used for names
\def \names{\"Ubungsblatt 8,\\ Datenbanksysteme II}

% content of right head area e.g. semester like WiSe 2012/13
\def \headRight{\today}

% set name for exercises
\def \exerciseName{Aufgabe}

\begin{document}
\exercise{}\\
\begin{itemize}
  \item Vereinigung ($\cup$)
  \begin{itemize}
    \item hashbasiert\\
		\begin{function}[H]
		\SetKwData{r}{r}\SetKwData{R}{R\textsubscript{1}}\SetKwData{S}{R\textsubscript{2}}\SetKwData{T}{R\textsubscript{res}}\SetKwData{partition}{partition}\SetKwData{H}{H}
		\SetKwFunction{Union}{union}\SetKwFunction{append}{append}\SetKwFunction{h}{h}
		\Function{\Union{\R, \S}}\\
		build hash table \H\;
		\ForEach{\tn{record} \r $\in$ \R}{\If{mismatch on \H via \h{\r}}{\append \r \KwTo \T\;}}
		\ForEach{\tn{record} \r $\in$ \S}{\If{mismatch on \H via \h{\r}}{\append \r \KwTo \T\;}}
		return \T\;
		\end{function}
    \item sortierungsbasiert\\
   		\begin{function}[H]
   		\SetKwData{r}{r}\SetKwData{ri}{r\textsubscript{i}}\SetKwData{rii}{r\textsubscript{i+1}}\SetKwData{R}{R\textsubscript{1}}\SetKwData{S}{R\textsubscript{2}}\SetKwData{T}{R\textsubscript{res}}\SetKwData{partition}{partition}\SetKwData{H}{H}
   		\SetKwFunction{Union}{union}\SetKwFunction{append}{append}\SetKwFunction{h}{h}
   		\Function{\Union{\R, \S}}\\
   		\ForEach{\tn{record} \r $\in$ \R}{\append \r \KwTo \T\;}
   		\ForEach{\tn{record} \r $\in$ \S}{\append \r \KwTo \T\;}
   		\ForEach{\tn{record} \ri $\in$ \T}{\If{\rii exists}{\If{\rii equals \ri}{\T $\leftarrow$ \T $\setminus$ \rii}}}
   		return \T\;
   		\end{function}
  \end{itemize}
  \newpage
  \item Vereinigung ($\cap$)
  \begin{itemize}
    \item hashbasiert\\
		\begin{function}[H]
		\SetKwData{r}{r}\SetKwData{R}{R\textsubscript{1}}\SetKwData{S}{R\textsubscript{2}}\SetKwData{T}{R\textsubscript{res}}\SetKwData{partition}{partition}\SetKwData{H}{H}
		\SetKwFunction{Union}{union}\SetKwFunction{append}{append}\SetKwFunction{h}{h}
		\Function{\Union{\R, \S}}\\
		build hash table \H\;
		\ForEach{\tn{record} \r $\in$ \R}{\If{mismatch on \H via \h{\r}}{\append \r \KwTo \T\;}}
		\ForEach{\tn{record} \r $\in$ \S}{\If{match on \H via \h{\r}}{\append \r \KwTo \T\;}}
		return \T\;
		\end{function}
    \item sortierungsbasiert\\
   		\begin{function}[H]
   		\SetKwData{r}{r}\SetKwData{ri}{r\textsubscript{i}}\SetKwData{rii}{r\textsubscript{i+1}}\SetKwData{R}{R\textsubscript{1}}\SetKwData{S}{R\textsubscript{2}}\SetKwData{T}{R\textsubscript{1,2}}\SetKwData{U}{R\textsubscript{res}}\SetKwData{partition}{partition}\SetKwData{H}{H}\SetKwData{riii}{r\textsubscript{res\textsubscript{|R\textsubscript{res}|}}}
   		\SetKwFunction{Union}{union}\SetKwFunction{append}{append}\SetKwFunction{h}{h}
   		\Function{\Union{\R, \S}}\\
   		\ForEach{\tn{record} \r $\in$ \R}{\append \r \KwTo \T\;}
   		\ForEach{\tn{record} \r $\in$ \S}{\append \r \KwTo \T\;}
   		
   		\ForEach{\tn{record} \ri $\in$ \T}{\If{\rii exists}{\If{\rii equals \ri}{\tcp{r\textsubscript{res\textsubscript{|R\textsubscript{res}|}} is the least entry of R\textsubscript{res}}\If{\riii $\neq$ \rii}{\U $\leftarrow$ \T $\setminus$ \ri}}}}
   		return \U\;
   		\end{function}
  \end{itemize}
\end{itemize}
%
\newpage 
%
\exercise{}\\
\begin{itemize}
	\item block nested loop join\\
		\begin{function}[H]
		\SetKwData{R}{R}\SetKwData{S}{S}\SetKwData{p}{p}\SetKwData{bR}{b\textsubscript{R}}\SetKwData{bS}{b\textsubscript{S}}\SetKwData{rbR}{r\textsubscript{b\textsubscript{R}}}\SetKwData{rbS}{r\textsubscript{b\textsubscript{S}}}
		\SetKwFunction{FFunction}{block\_nljoin}
		\Function{\FFunction{\R, \S, \p}}\\
		\ForEach{\bR-sized block in \R}{\ForEach{\bS-sized block in \S}{\tcc{performed in the buffer}
		\tcc{find matches in current \R- and \S-blocks and append them to the result}
		\ForEach{record \rbR $\in$ \bR}{\ForEach{record \rbS $\in$ \bS}{\If{$\langle\rbR,\rbS\rangle$ matches \p}{append $\langle\rbR,\rbS\rangle$ to the result and mark as matched}}\If{is not marked as copied}{append $\langle\rbR,-\rangle$ to the result}}}}
		\end{function}
		Tritt das Prädikat auf mindestens ein Tupel zusammengefügt aus beiden Relationen zu, wird dieses der Zieltabelle angehängt. Sollte sich kein solches Tupel für ein Tupel der Relation $R$ finden, wird $r_{b_R}$ ohne Partner der Zieltabelle angefügt, da es unmarkiert blieb.
	\item index nested loop join\\
   		\begin{function}[H]
		\SetKwData{R}{R}\SetKwData{S}{S}\SetKwData{p}{p}\SetKwData{r}{r}\SetKwData{s}{s}\SetKwData{R}{R}
		\SetKwFunction{FFunction}{index\_nljoin}\SetKwFunction{Search}{IndexScan}
		\Function{\FFunction{\R, \S, \p}}\\
		\ForEach{record \r $\in$ \R}{\tcc{scan \S-index using (key value in) \r and concatenate \r with all matching tuples s}
		\If{$\langle\r,\Search{\r, \S}\rangle$ matches \p}{\ForEach{\s $\in$ \Search{\r, \S}}{append $\langle\r,\s\rangle$ to result}\lElse{append $\langle\r,-\rangle$ to result}}}
   		\end{function}
   		Sofern der IndexScan mindestens ein Element liefert, finden sich für $r$ Partner. Andernfalls gibt es kein Tupel $\langle r, s\rangle$, $\langle r,-\rangle$ wird dem Ergebnis angefügt.
	\end{itemize}
%
\newpage
%
\exercise{}
%
\newpage
%
\exercise{}
\subExBegin{1{.}}
	\item
	\begin{enumerate}[(a)]
		\item \hfill\\
		\begin{function}[H]
		\SetKwData{R}{R}\SetKwData{S}{S}
		\SetKwFunction{FFunction}{open}
		\Function{\FFunction{}}\\
		\R.\FFunction{}\;
		\S.\FFunction{}\;
		\end{function}
		\begin{function}[H]
		\SetKwData{R}{R}\SetKwData{S}{S}
		\SetKwFunction{FFunction}{close}
		\Function{\FFunction{}}\\
		\R.\FFunction{}\;
		\S.\FFunction{}\;
		\end{function}
		\begin{function}[H]
		\SetKwData{R}{R}\SetKwData{S}{S}\SetKwData{r}{r}\SetKwData{s}{s}
		\SetKwFunction{FFunction}{next}
		\Function{\FFunction{}}\\
		\If{$($\r $\leftarrow$ \R.\FFunction{}$)$ $\neq$ $\langle \tn{EOF} \rangle$}{return \r\;
		\lElse{\If{$($\s $\leftarrow$ \S.\FFunction{}$)$ $\neq$ $\langle \tn{EOF} \rangle$}{return \s\;}\lElse{return $\langle \tn{EOF} \rangle$}}}
		\end{function}
		\newpage
		\item \hfill\\
		\begin{function}[H]
		\SetKwData{R}{R}\SetKwData{S}{S}
		\SetKwFunction{FFunction}{open}
		\Function{\FFunction{}}\\
		\R.\FFunction{}\;
		\S.\FFunction{}\;
		\end{function}
		\begin{function}[H]
		\SetKwData{R}{R}\SetKwData{S}{S}
		\SetKwFunction{FFunction}{close}
		\Function{\FFunction{}}\\
		\R.\FFunction{}\;
		\S.\FFunction{}\;
		\end{function}
		\begin{function}[H]
		\SetKwData{R}{R}\SetKwData{S}{S}\SetKwData{r}{r}\SetKwData{s}{s}\SetKwData{H}{H}
		\SetKwFunction{FFunction}{next}\SetKwFunction{h}{h}
		\Function{\FFunction{}}\\
		\If{$(($\r $\leftarrow$ \R.\FFunction{}$)$ $\neq$ $\langle \tn{EOF} \rangle)$ $\&$ \r mismatches \H}{add \r via \h{r} to \H\;return \r\;
		\lElse{\If{$(($\s $\leftarrow$ \S.\FFunction{}$)$ $\neq$ $\langle \tn{EOF} \rangle)$ $\&$ \s mismatches \H}{add \s via \h{r} to \H\;return \s\;}\lElse{\If{$(($\r $\leftarrow$ \R.\FFunction{}$)$ $\neq$ $\langle \tn{EOF} \rangle)$ $\&$ $($\s $\leftarrow$ \S.\FFunction{}$)$ $\neq$ $\langle \tn{EOF} \rangle))$}{return \FFunction{}\;}}\lElse{return $\langle \tn{EOF} \rangle$}}}
		\end{function}
		\newpage
		\item \hfill\\
		\begin{function}[H]
		\SetKwData{R}{R}\SetKwData{S}{S}
		\SetKwFunction{FFunction}{open}
		\Function{\FFunction{}}\\
		\R.\FFunction{}\;
		\S.\FFunction{}\;
		\end{function}
		\begin{function}[H]
		\SetKwData{R}{R}\SetKwData{S}{S}
		\SetKwFunction{FFunction}{close}
		\Function{\FFunction{}}\\
		\R.\FFunction{}\;
		\S.\FFunction{}\;
		\end{function}
		\begin{function}[H]
		\SetKwData{R}{R}\SetKwData{S}{S}\SetKwData{r}{r}\SetKwData{s}{s}\SetKwData{sum}{sum}
		\SetKwFunction{FFunction}{next}
		\Function{\FFunction{}}\\
		\While{$($\r $\leftarrow$ \R.\FFunction{}$)$ $\neq$ $\langle \tn{EOF} \rangle$}{\sum $\leftarrow$ \sum + \r}
		\While{$($\s $\leftarrow$ \S.\FFunction{}$)$ $\neq$ $\langle \tn{EOF} \rangle$}{\sum $\leftarrow$ \sum + \s}
		return $\langle \tn{EOF} \rangle$\;
		\end{function}
	\end{enumerate}
	\item \code{UNION ALL} und \code{UNION} sind nicht blockende Operatoren, mit \code{next()} wird nur ein einziges Tupel geliefert.\\
	\code{SUM} hingegen muss alle Tupel durchlaufen, da sonst unvollständig, er ist ein blockender Operator.
\subExEnd{}


\end{document}