\documentclass{uni_tue_template}

\usepackage{color}
\usepackage[colorlinks]{hyperref}
\definecolor{darkred}{rgb}{0.5,0,0}
\definecolor{darkgreen}{rgb}{0,0.5,0}
\definecolor{darkblue}{rgb}{0,0,0.7}
\hypersetup{
			linkcolor={darkblue},
			filecolor={darkgreen},
			urlcolor={darkred},
			citecolor={darkblue},
}

\usepackage{pgf}
\usepackage{tikz}
\usetikzlibrary{arrows}
\tikzset{<->,>=stealth',shorten >=1pt,auto,node distance=4.5cm,bend angle=20,
		 every path/.style={font=\ttfamily\scriptsize\bfseries},
		 relation/.style={
		 	circle,fill=blue!20,draw,font=\ttfamily\footnotesize\bfseries,
			prefix after command= {\pgfextra{\tikzset{every label/.style={font=\ttfamily\footnotesize\bfseries,yshift=-1cm}}}}
			 }
}
\usepackage{sistyle}

\lstdefinestyle{sql}{%
	language=SQL,%
	commentstyle=\color{comment},%
	stringstyle=\color{string},%
	keywordstyle=\color{keyword}\bfseries,%
	morekeywords={text, real, begin},%
	basicstyle=\ttfamily\footnotesize,%
	numbers=left,
	numberstyle=\tiny\color{black},
	stepnumber=1,
	showstringspaces=true,%
	columns=fixed,%
	moredelim=[is][\itshape]{@@}{@@},%
	tabsize=2
}
\lstset{aboveskip=-.05cm,style=sql, firstline=4}

\newcommand{\otext}[1]{\overset{\mathclap{\rule[-.3\baselineskip]{0pt}{0cm}\textmd{#1}}}}
\newcommand{\set}[1]{\mathbb{#1}}
\newcommand{\code}[1]{\texttt{{\footnotesize #1}}}
\newcommand{\md}[1]{\textmd{#1}}

% content of left head area e.g. subject like ETI
\def \headLeft{Andreas Schmied (3087156),\newline Tobias Stumpp (3798377)}

% content of center head area, used for names
\def \names{\"Ubungsblatt 2,\\ Datenbanksysteme II}

% content of right head area e.g. semester like WiSe 2012/13
\def \headRight{\today}

% set name for exercises
\def \exerciseName{Aufgabe}

\begin{document}
\exercise{}
\subExBegin{1{.}}
	\item
	\begin{enumerate}
		\item[FIFO]
		\begin{enumerate}
			\item[Buffergröße 3 Frames]\hfill\\
			$\begin{array}{r|ccc|l}
			$victim$ &   1 &   2 &   3 & $page$	 \\
			\hline
					 &	 &	 & p_1 & $miss $ p_1\\
					 &	 & p_1 & p_2 & $miss $ p_2\\
					 & p_1 & p_2 & p_3 & $miss $ p_3\\
				 p_1 & p_2 & p_3 & p_4 & $miss $ p_4\\
				 p_2 & p_3 & p_4 & p_1 & $miss $ p_1\\
				 p_3 & p_4 & p_1 & p_2 & $miss $ p_2\\
				 p_4 & p_1 & p_2 & p_5 & $miss $ p_5\\
					 & p_1 & p_2 & p_5 & $hit $ p_1 \\
					 & p_1 & p_2 & p_5 & $hit $ p_2 \\
				 p_1 & p_2 & p_5 & p_3 & $miss $ p_3\\
				 p_2 & p_5 & p_3 & p_4 & $miss $ p_4\\
					 & p_5 & p_3 & p_4 & $hit $  p_4
			\end{array}$
			\item[Buffergröße 4 Frames]\hfill\\
			$\begin{array}{r|cccc|l}
			$victim$ &   1 &   2 &   3 &   4 & $page$	 \\
			\hline
					 &	 &	 &	 & p_1 & $miss $ p_1\\
					 &	 &	 & p_1 & p_2 & $miss $ p_2\\
					 &	 & p_1 & p_2 & p_3 & $miss $ p_3\\
					 & p_1 & p_2 & p_3 & p_4 & $miss $ p_4\\
					 & p_1 & p_2 & p_3 & p_4 & $hit $ p_1 \\
					 & p_1 & p_2 & p_3 & p_4 & $hit $ p_2 \\
				 p_1 & p_2 & p_3 & p_4 & p_5 & $miss $ p_5\\
				 p_2 & p_3 & p_4 & p_5 & p_1 & $miss $ p_1\\
				 p_3 & p_4 & p_5 & p_1 & p_2 & $miss $ p_2\\
				 p_4 & p_5 & p_1 & p_2 & p_3 & $miss $ p_3\\
				 p_5 & p_1 & p_2 & p_3 & p_4 & $miss $ p_4\\
				 p_1 & p_2 & p_3 & p_4 & p_5 & $miss $ p_4
			\end{array}$
		\end{enumerate}
		\code{FIFO} löst mit $3$ Frames $9$ page-faults aus, mit $4$ Frames entsprechend $10$ page-faults.
		\newpage
		\item[LRU]
		\begin{enumerate}
			\item[Buffergröße 3 Frames]\hfill\\
			$\begin{array}{r|ccc|l}
			$victim$ &   1 &   2 &   3 & $page$	 \\
			\hline
					 &	 &	 & p_1 & $miss $ p_1\\
					 &	 & p_1 & p_2 & $miss $ p_2\\
					 & p_1 & p_2 & p_3 & $miss $ p_3\\
				 p_1 & p_2 & p_3 & p_4 & $miss $ p_4\\
				 p_2 & p_3 & p_4 & p_1 & $miss $ p_1\\
				 p_3 & p_4 & p_1 & p_2 & $miss $ p_2\\
				 p_4 & p_1 & p_2 & p_5 & $miss $ p_5\\
					 & p_2 & p_5 & p_1 & $hit $ p_1 \\
					 & p_5 & p_1 & p_2 & $hit $ p_2 \\
				 p_5 & p_1 & p_2 & p_3 & $miss $ p_3\\
				 p_1 & p_2 & p_3 & p_4 & $miss $ p_4\\
				 p_2 & p_3 & p_4 & p_5 & $miss $ p_5
			\end{array}$
			\item[Buffergröße 4 Frames]\hfill\\
			$\begin{array}{r|cccc|l}
			$victim$ &   1 &   2 &   3 &   4 & $page$	 \\
			\hline
					 &	 &	 &	 & p_1 & $miss $ p_1\\
					 &	 &	 & p_1 & p_2 & $miss $ p_2\\
					 &	 & p_1 & p_2 & p_3 & $miss $ p_3\\
					 & p_1 & p_2 & p_3 & p_4 & $miss $ p_4\\
					 & p_2 & p_3 & p_4 & p_1 & $hit $ p_1 \\
					 & p_3 & p_4 & p_1 & p_2 & $hit $ p_2 \\
				 p_3 & p_4 & p_1 & p_2 & p_5 & $miss $ p_5\\
					 & p_4 & p_2 & p_5 & p_1 & $hit $ p_1 \\
					 & p_4 & p_5 & p_1 & p_2 & $hit $ p_2 \\
				 p_4 & p_5 & p_1 & p_2 & p_3 & $miss $ p_3\\
				 p_5 & p_1 & p_2 & p_3 & p_4 & $miss $ p_4\\
				 p_1 & p_2 & p_3 & p_4 & p_5 & $miss $ p_5
			\end{array}$
		\end{enumerate}
		\code{LRU} löst mit $3$ Frames $10$ page-faults aus, mit $4$ Frames entsprechend $8$ page-faults.
	\end{enumerate}
	\item Die Information über die Häufigkeit der bisherigen Zugriffe auf die einzelnen Seiten, also die Relevanz einer Seite gegenüber anderen Seiten.
\subExEnd
%
\newpage
%
\exercise{}
\subExBegin{1{.}}
	\item \code{LRU-1} entspricht dem Verhalten von \code{LRU}, da $b_t(p, k)$ das Durchsuchen des Referenzstrings $r$ nach einer spezifizierten Seite beim ersten Fund abbricht und folglich die Distanz der im String am weitesten zurückliegenden Seite am Größten ist, was wie beim \code{LRU} einem Victim der "least recent" page entspricht.
	\item Misses bestimmt durch Simulation, (src und jar sind beigefügt \code{LRUk.jar})\\
	Seitenfehler je \code{LRU-k}:
	\begin{itemize}
		\item[\code{LRU-1}:] 189
		\item[\code{LRU-2}:] 100
	\end{itemize}
	\begin{itemize}
		\item[Vorteile  \code{LRU-2}:] deutlich schneller als \code{LRU}
		\item[Nachteile \code{LRU-2}:] \code{LRU} reicht es aus Information über die gepufferten Pages zu haben, \code{LRU-2} benötigt die weitere Information zum vorherigen Verlauf (Referensstring \code{R}).
	\end{itemize} 
\subExEnd
%
\newpage
%
\exercise{}
\subExBegin{1{.}}
	\item \hfill\\
		\begin{enumerate}
			\item[über Tupel]\hfill\\
				Record:\\
				$\SI{4}{Byte} + \SI{30}{Byte} + \SI{2}{Byte} + \SI{4}{Byte} = \SI{40}{Byte}$
				Freier Speicher:\\
				$\SI{1024}{Byte} - \SI{68}{Byte} - \SI{40}{Byte} - \SI{1}{Byte} = \SI{915}{Byte}$
			\item[über Spalten]\hfill\\
				Record:\\
				$p_0$: $\SI{4}{Byte}$\\
				$p_1$: $\SI{30}{Byte}$\\
				$p_2$: $\SI{2}{Byte}$\\
				$p_3$: $\SI{4}{Byte}$\\
				Freier Speicher:\\
				$p_0$: $\SI{1024}{Byte} - \SI{68}{Byte} - \SI{4}{Byte} - \SI{1}{Byte} = \SI{951}{Byte}$\\
				$p_1$: $\SI{1024}{Byte} - \SI{68}{Byte} - \SI{30}{Byte} - \SI{1}{Byte} = \SI{925}{Byte}$\\
				$p_2$: $\SI{1024}{Byte} - \SI{68}{Byte} - \SI{2}{Byte} - \SI{1}{Byte} = \SI{953}{Byte}$\\
				$p_3$: $\SI{1024}{Byte} - \SI{68}{Byte} - \SI{4}{Byte} - \SI{1}{Byte} = \SI{951}{Byte}$\\
				Gesamt: $\SI{3780}{Byte}$
		\end{enumerate}
		\newpage
	\item \hfill\\
		\begin{enumerate}
			\item[über Tupel]\hfill\\
				Frei: $\SI{1024}{Byte} - \SI{68}{Byte} - 23 \cdot \SI{40}{Byte} - \SI{3}{Byte} = \SI{33}{Byte}$\\
				Für beide Queries werde alle Pages referenziert, es gilt:
				$\left\lceil\dfrac{100000}{23}\right\rceil = \SI{4348}{Pages}$
			\item[über Spalten]\hfill\\
				\begin{enumerate}
					\item[PNr, Gehalt:]\hfill\\
						Frei: $\SI{1024}{Byte} - \SI{68}{Byte} - 231 \cdot \SI{4}{Byte} - \SI{29}{Byte} = \SI{3}{Byte}$
					\item[Name:]\hfill\\
						Frei: $\SI{1024}{Byte} - \SI{68}{Byte} - 31 \cdot 30 - 4 = \SI{22}{Byte}$
					\item[Rang:]\hfill\\
						Frei: $\SI{1024}{Byte} - \SI{68}{Byte} - 449 \cdot 2 - \SI{57}{Byte} = \SI{1}{Byte}$
				\end{enumerate}
				\begin{enumerate}[(a)]
					\item\hfill\\
						\begin{enumerate}
							\item[Pnr:]
								$\left\lceil\dfrac{100000}{231}\right\rceil = 433$
							\item[Name:]
								$\left\lceil\dfrac{100000}{31}\right\rceil = 3226$
 							\item[Rang:]
 								$\left\lceil\dfrac{100000}{449}\right\rceil = 228$
							\item[Gehalt:]
								$\left\lceil\dfrac{100000}{231}\right\rceil = 433$
								Gesamt: $\SI{4320}{Pages}$
						\end{enumerate}
					\item\hfill\\
						\begin{enumerate}
							\item[Name:]
								 $\left\lceil\dfrac{100000}{31}\right\rceil = \SI{3226}{Pages}$ pages
						\end{enumerate}
				\end{enumerate}
		\end{enumerate}
		\newpage
	\item Die Unterschiede in der Organisation von Daten lassen sich maßgeblich auf Festplattenzugriffe zurückführen, also wieviele Pages geladen werden.\\
	Daher ergibt sich bei Organisation über\\
		\begin{enumerate}
			\item[Tupel] Effizient bei Zugriff auf viele Spalten bei wenigen Tupeln oder beim Ändern von Tupeln.
			\item[Spalten] Effizient bei Zugriff auf viele Tupel bei wenigen Spalten oder beim Ändern von Spalten.
		\end{enumerate}
\subExEnd

\end{document}