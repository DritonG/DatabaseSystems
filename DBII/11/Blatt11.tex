\documentclass{uni_tue_template}

\usepackage{color}
\usepackage[colorlinks]{hyperref}
\definecolor{darkred}{rgb}{0.5,0,0}
\definecolor{darkgreen}{rgb}{0,0.5,0}
\definecolor{darkblue}{rgb}{0,0,0.7}
\hypersetup{
			linkcolor={darkblue},
			filecolor={darkgreen},
			urlcolor={darkred},
			citecolor={darkblue},
}

\usepackage[export]{adjustbox}
\usepackage{tikz}
\usetikzlibrary{shapes}
\usepackage{rotating}
\usepackage{sistyle}

\lstdefinestyle{sql}{%
	language=SQL,%
	commentstyle=\color{comment},%
	stringstyle=\color{string},%
	keywordstyle=\color{keyword}\bfseries,%
	morekeywords={text, real, begin},%
	basicstyle=\ttfamily\footnotesize,%
	numbers=left,
	numberstyle=\tiny\color{black},
	stepnumber=1,
	showstringspaces=true,%
	columns=fixed,%
	moredelim=[is][\itshape]{@@}{@@},%
	tabsize=2
}
\lstset{aboveskip=-.6cm,style=sql, firstline=4}

\newcommand{\otext}[1]{\overset{\mathclap{\rule[-.3\baselineskip]{0pt}{0cm}\textmd{#1}}}}
\newcommand{\set}[1]{\mathbb{#1}}
\newcommand{\code}[1]{\texttt{{\footnotesize #1}}}
\newcommand{\md}[1]{\textmd{#1}}
\newcommand{\tn}[1]{\textup{\textsf{#1}}}
\newcommand{\ti}[1]{\textit{\textsf{#1}}}

% content of left head area e.g. subject like ETI
\def \headLeft{Andreas Schmied (3087156),\newline Tobias Stumpp (3798377)}

% content of center head area, used for names
\def \names{\"Ubungsblatt 11,\\ Datenbanksysteme II}

% content of right head area e.g. semester like WiSe 2012/13
\def \headRight{\today}

% set name for exercises
\def \exerciseName{Aufgabe}

\begin{document}
\exercise{2}
\subExBegin{{.}}
	\item
	\begin{enumerate}[(a)]
		\item \hfill\\
		\begin{enumerate}[i.]
			\item \hfill\\
\begin{lstlisting}
SELECT Shipment.sno
FROM   Shipment, Part
WHERE  Shipment.pno = Part.pno AND
       Part.price > 100
\end{lstlisting}
			\item \hfill\\
\begin{lstlisting}
SELECT Shipment.sno
FROM   Shipment, Project
WHERE  Shipment.pno = Part.pno AND
       Shipment.jno = Project.jno AND
       Project.jloc = 'Philadelphia'
\end{lstlisting}
		\end{enumerate}
		\item \hfill\\
		\begin{enumerate}[i.]
			\item Ein vereinfachen des Joins würde zum Kreuzprodukt führen, ebenso würde ein Abändern der Selektion zu einem anderen Ergebnis führen.\\ Damit ist keine Vereinfachung möglich.
			\item Ein \code{Projekt} ist einem eindeutigen \code{Part} zugeschrieben, wogegen ein \code{Part} mehreren \code{Projekt} zugewiesen sein kann.\\
			Eine gleich resultierende Vereinfachung ist demnach:\\
\begin{lstlisting}
SELECT Shipment.sno
FROM   Shipment, Project
WHERE  Shipment.jno = Project.jno AND
       Project.jloc = 'Philadelphia'
\end{lstlisting}
wobei nachfolgende Abfrage zusätzlich Tupel liefern würde, die dem richtigen \code{Part}, aber einem falschen \code{Projekt} zugewiesen sind:\\
\begin{lstlisting}
SELECT Shipment.sno
FROM   Shipment, Project
WHERE  Shipment.pno = Part.pno AND
       Project.jloc = 'Philadelphia'
\end{lstlisting}
		\end{enumerate}
	\end{enumerate}
	\newpage
	\item
	\begin{enumerate}[(a)]
		\item \hfill\\
		\begin{enumerate}[i.]
			\item nested\\
			\begin{tabular}{|c|c|}
			\multicolumn{2}{c}{\textbf{Subquery}}\\
			\hline
			\code{Supply.pnum = Parts.pnum}&\code{COUNT(shipdate)}\\
			\hline
			3&2\\
			10&1\\
			8&0\\
			\hline
			\end{tabular}
			\begin{tabular}{|c|}
			\multicolumn{1}{c}{\textbf{Result}}\\
			\hline
			\code{pnum}\\
			\hline
			10\\
			8\\
			\hline
			\end{tabular}
			\item unnested\\
			\begin{tabular}{|c|c|}
			\multicolumn{2}{c}{\textbf{Subquery}}\\
			\hline
			\code{Supply.pnum}&\code{COUNT(shipdate) AS ct}\\
			\hline
			3&2\\
			10&1\\
			\hline
			\end{tabular}
			\begin{tabular}{|c|}
			\multicolumn{1}{c}{\textbf{Result}}\\
			\hline
			\code{pnum}\\
			\hline
			10\\
			\hline
			\end{tabular}
		\end{enumerate}
		\item \hfill\\
		\begin{enumerate}[i.]
			\item \hfill\\
\begin{lstlisting}
SELECT          pnum
FROM            Parts
LEFT OUTER JOIN (SELECT pnum, COUNT(shipdate) AS ct
                 FROM   Supply
                 WHERE  shipdate < 1-1-80
                 GROUP BY pnum) T ON (Parts.pnum = T.pnum)
WHERE           (Parts.qoh = 0 AND 
                T.count IS NULL) OR
                Parts.qoh = T.ct;
\end{lstlisting}
		Tupel zu denen das \code{GROUP} subquery kein Ergebnis liefert werden mit dem \code{LEFT OUTER JOIN} mit \code{NULL} in den Join aufgenommen. Die \code{WHERE} clause nimmt Selektiert diese Ergebnisse zusätzlich zu der bisherigen Selektion.
		\end{enumerate}
	\end{enumerate}
\subExEnd{}


\end{document}