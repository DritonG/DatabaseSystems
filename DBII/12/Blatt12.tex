\documentclass{uni_tue_template}

\usepackage{color}
\usepackage[colorlinks]{hyperref}
\definecolor{darkred}{rgb}{0.5,0,0}
\definecolor{darkgreen}{rgb}{0,0.5,0}
\definecolor{darkblue}{rgb}{0,0,0.7}
\hypersetup{
			linkcolor={darkblue},
			filecolor={darkgreen},
			urlcolor={darkred},
			citecolor={darkblue},
}

\usepackage[export]{adjustbox}
\usepackage{tikz}
\usetikzlibrary{shapes}
\usepackage{rotating}
\usepackage{sistyle}

\lstdefinestyle{sql}{%
	language=SQL,%
	commentstyle=\color{comment},%
	stringstyle=\color{string},%
	keywordstyle=\color{keyword}\bfseries,%
	morekeywords={text, real, begin},%
	basicstyle=\ttfamily\footnotesize,%
	numbers=left,
	numberstyle=\tiny\color{black},
	stepnumber=1,
	showstringspaces=true,%
	columns=fixed,%
	moredelim=[is][\itshape]{@@}{@@},%
	tabsize=2
}
\lstset{aboveskip=-.6cm,style=sql, firstline=4}

\newcommand{\otext}[1]{\overset{\mathclap{\rule[-.3\baselineskip]{0pt}{0cm}\textmd{#1}}}}
\newcommand{\set}[1]{\mathbb{#1}}
\newcommand{\code}[1]{\texttt{{\footnotesize #1}}}
\newcommand{\md}[1]{\textmd{#1}}
\newcommand{\tn}[1]{\textup{\textsf{#1}}}
\newcommand{\ti}[1]{\textit{\textsf{#1}}}

% content of left head area e.g. subject like ETI
\def \headLeft{Andreas Schmied (3087156),\newline Tobias Stumpp (3798377)}

% content of center head area, used for names
\def \names{\"Ubungsblatt 12,\\ Datenbanksysteme II}

% content of right head area e.g. semester like WiSe 2012/13
\def \headRight{\today}

% set name for exercises
\def \exerciseName{Aufgabe}

\begin{document}
\exercise{}\\
  \begin{tabular}{l|l|l}
  $\mathcal{T}_1$&$\mathcal{T}_2$&$\mathcal{T}_3$\\
  \hline
  $\cdot$&$\cdot$&\code{sharedLock(X)}\\
  $\cdot$&$\cdot$&\code{r(X)}\\
  \code{sharedLock(X)}&$\cdot$&$\cdot$\\
  \code{r(X)}&$\cdot$&$\cdot$\\
  $\cdot$&\code{sharedLock(Y)}&$\cdot$\\
  $\cdot$&\code{r(Y)}&$\cdot$\\
  \code{exclusiveLock(Y)}&$\cdot$&$\cdot$\\
  \code{w(Y)}&$\cdot$&$\cdot$\\
  \code{unlock(X)}&$\cdot$&$\cdot$\\
  \code{unlock(Y)}&$\cdot$&$\cdot$\\
  $\cdot$&\code{sharedLock(X)}&$\cdot$\\
  $\cdot$&\code{r(X)}&$\cdot$\\
  $\cdot$&\code{unlock(X)}&$\cdot$\\
  $\cdot$&\code{unlock(Y)}&$\cdot$\\
  $\cdot$&$\cdot$&\code{exclusiveLock(X)}\\
  $\cdot$&$\cdot$&\code{w(X)}\\
  $\cdot$&$\cdot$&\code{unlock(X)}\\
  \code{c}&$\cdot$&$\cdot$\\
  $\cdot$&\code{c}&$\cdot$\\
  $\cdot$&$\cdot$&\code{c}
  \end{tabular}\\
  Dieses Szenario erzeugt nach \code{2PL} Regeln das gefragte Schedule.
%
\newpage
%
\exercise{}\\
\subExBegin{1{.}}
  \item \hfill\\
  \begin{align*}
  S_1 &= \left< rl_3(Z), r_3(Z), rl_2(X), r_2(X), rl_2(Y), r_2(Y), wl_2(X), w_2(X), ul_2(X), ul_2(Y), c_2, \right.\\
  &\phantom{{}=1}\left. wl_3(Z), w_3(Z), rl_1(X), r_1(X), wl_1(Y), w_1(Y), ul_1(X), ul_1(Y), c_1, rl_3(X), r_3(X), \right.\\
  &\phantom{{}=1}\left. ul_3(X), ul_3(Z), c_3 \right>\\
  S_2 &= \left< rl_1(X), r_1(X), rl_3(W), r_3(W), rl_2(Y), r_2(Y), wl_2(X), w_2(X), wl_3(Y), w_3(Y), \right.\\
  &\phantom{{}=1}\left. rl_3(Z), r_3(Z), ul_3(W), ul_3(X), ul_3(Z), c_3, wl_1(Z), w_1(Z), ul_1(W), ul_1(X), c_1, \right.\\
  &\phantom{{}=1}\left. rl_2(W), r_2(W), ul_2(W), ul_2(X), ul_2(Y), c_2 \right>
  \intertext{Mit $S_2$ folgt eine Deadlock! Siehe}
  &wl_2(X) \code{\text{ denied - }X\text{ is locked by }$\mathcal{T}_1$},\\
  &[\ldots],\\
  &wl_3(Y) \code{\text{ denied - }Y\text{ is locked by }$\mathcal{T}_2$},\\
  &[\ldots],\\
  &\:wl_1(W) \code{\text{ denied - }W\text{ is locked by }$\mathcal{T}_3$}
  \intertext{}
  S_3 &= \left< rl_1(Y), r_1(Y), rl_3(X), r_3(X), wl_3(X), w_3(X), rl_2(X), r_2(X), wl_1(x), w_1(Y), \right.\\
  &\phantom{{}=1}\left. ul_1(Y), c_1, rl_3(Y), r_3(Y), wl_2(Y), w_2(Y), ul_2(X), ul_2(Y), c_2, rl_3(Z), r_3(Z), \right.\\
  &\phantom{{}=1}\left. ul_3(X), ul_3(Y), ul_3(Z), c_3 \right>
  \end{align*}
  \item \hfill\\
  Nicht konservativ?
\subExEnd{}

\end{document}